%-------------------------------------------------------------------------------
%\usepackage{amssymb} % Plus de symboles mathématiques.
%\usepackage{amsfonts} % Plus de polices mathématiques.
\usepackage[nice]{nicefrac} % Plus belles fractions.
\usepackage{braket} % Notation de Dirac.
\usepackage{bbm} % Police mathématique en gras.
%\usepackage{mathrsfs} % Police mathématique cursive.
\usepackage{esint} % Plus d'intégrales
\usepackage{cancel} % Permet de barrer des termes.
\usepackage{mathtools} % Ajustements du mode mathématique.
\usepackage{slashed} % Barrer des caractère uniques.
\usepackage{xargs} % Permet de mieux formater les argument supplémentaires des commandes.
%-------------------------------------------------------------------------------
%\usepackage{lmodern} % Polices d'écritures.
%\usepackage{feyn} % Diagrammes de Feynman dans des environnement d'équations.

%%%%%%%%%%%%%%%%%%%%%%%%%%%
% Mathématiques et physique
%%%%%%%%%%%%%%%%%%%%%%%%%%%%
% Unités internationales -----------------------
\newcommand{\ampere}{\text{A}}
\newcommand{\bell}{\text{B}}
\newcommand{\celsius}{\degree\text{C}}
\newcommand{\coulomb}{\text{C}}
\newcommand{\degree}{\,^{\circ}}
\newcommand{\farad}{\text{F}}
\newcommand{\electro}{\text{e}}
\newcommand{\gram}{\text{g}}
\newcommand{\henry}{\text{H}}
\newcommand{\hertz}{\text{Hz}}
\newcommand{\hour}{\text{h}}
\newcommand{\joule}{\text{J}}
\newcommand{\kelvin}{\text{K}}
\newcommand{\meter}{\text{m}}
\newcommand{\minute}{\text{m}}
\newcommand{\mole}{\text{mol}}
\newcommand{\newton}{\text{N}}
\newcommand{\ohm}{\Omega}
\newcommand{\pascal}{\text{Pa}}
\newcommand{\rad}{\text{rad}}
\newcommand{\second}{\text{s}}
\newcommand{\tesla}{\text{T}}
\newcommand{\torr}{\text{Torr}}
\newcommand{\volt}{\text{V}}
\newcommand{\watt}{\text{W}}
%
\newcommand{\tera}{\text{T}}
\newcommand{\giga}{\text{G}}
\newcommand{\mega}{~\text{M}}
\newcommand{\kilo}{~\text{k}}
\newcommand{\deci}{\text{d}}
\newcommand{\centi}{\text{c}}
\newcommand{\milli}{\text{m}}
\newcommand{\micro}{\mu}
\newcommand{\nano}{\text{n}}
\newcommand{\pico}{\text{p}}
\newcommand{\femto}{\text{f}}
%
\newcommand{\units}[1]{\text{#1}}
\newcommand{\tothe}[1]{\textsuperscript{#1}}
%
\newcommand{\per}{\text{/}}
%
\newcommand{\Time}[3]{#1\hour~#2\minute~#3\second} % ARGUMENTS OPTIONNELS!!!!!!!
\newcommand{\Angle}[3]{#1^{\circ}~#2'~#3''} % ARGUMENTS OPTIONNELS!!!!!!


% Plus bel epsilon -----------------------
\let\oldepsilon\epsilon
\let\epsilon\varepsilon
\let\varepsilon\oldepsilon


% Plus belle \bar -----------------------
\renewcommand{\bar}[1]{\mkern 1.5mu\overline{\mkern-1.5mu#1\mkern-1.5mu}\mkern 1.5mu}


% Équations -----------------------
\newcommand{\al}[1]{\begin{align} #1 \end{align}} % Équation(s) numérotée(s)
\newcommand{\eqn}[1]{\begin{align*} #1 \end{align*}} % Équation(s) non-numérotée(s)
\newcommand{\sys}[1]{\begin{dcases*} #1 \end{dcases*}} % Système d'équation avec un crochet à gauche


% Exposants -----------------------
\newcommand{\Exp}[1]{\text{e}^{#1}}		% e^#
\newcommand{\E}[1]{\times 10^{#1}}		% X 10^#


% Délimitations -----------------------
\newcommand{\p}[1]{\left( #1 \right)}	% (#)
\newcommand{\cro}[1]{\left[ #1 \right]}	% [#]
\newcommand{\abs}[1]{\left| #1\right|}	% |#|
\newcommand{\avg}[1]{\left\langle #1 \right\rangle} % <#>
\newcommand{\acc}[1]{\left\lbrace #1 \right\rbrace} % {#}
\newcommand{\com}[2]{\left[#1,#2\right]} % Commutateur
\newcommand{\poi}[2]{\left\{#1,#2\right\}} % Crochets de poisson


% Vecteurs -----------------------
\newcommand{\ve}[1]{\mathbf{#1}} % Vecteurs en gras
\newcommand{\vu}[1]{\hat{\ve{#1}}} % Vecteurs unitaire en gras
\newcommand{\tens}{\otimes} % Produit tensoriel/de Kronecker
\newcommand{\sumdir}{\oplus} % Somme directe
\newcommand{\nablav}{\bm{\nabla}} % Gradient vectoriel


% Rapport trigonométriques avec formatage automatique  -----------------------
\newcommandx{\Sin}[2][1={}]{\text{sin}^{#1}\!\p{#2}}
\newcommandx{\Cos}[2][1={}]{\text{cos}^{#1}\!\p{#2}}
\newcommandx{\Tan}[2][1={}]{\text{tan}^{#1}\!\p{#2}}
\newcommandx{\Csc}[2][1={}]{\text{csc}^{#1}\!\p{#2}}
\newcommandx{\Sec}[2][1={}]{\text{sec}^{#1}\!\p{#2}}
\newcommandx{\Cot}[2][1={}]{\text{cot}^{#1}\!\p{#2}}
\newcommandx{\Arcsin}[2][1={}]{\text{arcsin}^{#1}\!\p{#2}}
\newcommandx{\Arccos}[2][1={}]{\text{arccos}^{#1}\!\p{#2}}
\newcommandx{\Arctan}[2][1={}]{\text{arctan}^{#1}\!\p{#2}}
\newcommandx{\Sinh}[2][1={}]{\text{sinh}^{#1}\!\p{#2}}
\newcommandx{\Cosh}[2][1={}]{\text{cosh}^{#1}\!\p{#2}}
\newcommandx{\Tanh}[2][1={}]{\text{tanh}^{#1}\!\p{#2}}


% Matrices -----------------------
\newcommand{\mat}[1]{\begin{bmatrix} #1 \end{bmatrix}} % Matrice avec crochets.
\newcommand{\pmat}[1]{\begin{pmatrix} #1 \end{pmatrix}} % Matrice avec parenthèses.
\newcommand{\deter}[1]{\abs{\begin{matrix} #1 \end{matrix}}} % Déterminant sous forme matriciel.
\newcommandx{\mO}[2][1={}, 2={}]{ \def\temp{#2}\ifx\temp\empty\ve{O}_{#1}\else\ve{O}_{#1\times #2}\fi}% Matrice Nulle.
\newcommandx{\mI}[2][1={}, 2={}]{ \def\temp{#2}\ifx\temp\empty\ve{I}_{#1}\else\ve{O}_{#1\times #2}\fi}% Matrice identité.
\newcommand{\Det}[1]{\text{det}\p{#1}} % det(#)
\newcommand{\Tr}[1]{\text{Tr}\p{#1}} % Tr(#)


% Dérivées -----------------------
\newcommand{\D}{\text{d}} % "d" d'une différentielle
\newcommandx{\DD}[3][1={},3={}]{\frac{\text{D}^{#3}#1}{\text{D}{#2}^{#3}}} % Dérivée Lagrangienne
\newcommandx{\dd}[3][1={},3={}]{\frac{\D^{#3}#1}{\D{#2}^{#3}}} % Dérivée totale selon #2, #1 est la fonction est #3 est l'ordre.
\newcommand{\del}{\partial} % Del de dérivée partielle
\newcommandx{\ddp}[3][1={},3={}]{\frac{\del^{#3}#1}{\del{#2}^{#3}}} % Dérivée partielle selon #2, #1 est la fonction est #3 est l'ordre.
\newcommand{\eval}[1]{\left. {#1} \right|} % Barre à la droite d'une expression pour évaluation
\newcommand{\delbar}{\slashed{\del}} % Différentielle partielle inexacte.
\newcommand{\dbar}{\dj}% Différentielle inexacte


% Intégrales -----------------------
\newcommand{\intinf}{\int\displaylimits_{-\infty}^{\infty}}
\newcommandx{\Int}[2][1={},2={}]{\int\displaylimits_{#1}^{#2}}


% Opérations -----------------------
\newcommand{\fourier}[1]{\mathcal{F}\acc{#1}} % Transformée de Fourier F{#}
\newcommand{\fourieri}[1]{\mathcal{F}^{-1}\acc{#1}} % Transformée de Fourier inverse F^-1{#}
\renewcommand{\Re}[1]{\text{Re}\acc{#1}} % Re{#}
\renewcommand{\Im}[1]{\text{Im}\acc{#1}} % Im{#}


% Ensembles -----------------------
\newcommand{\N}{\mathbbm{N}} % Naturels
\newcommand{\Z}{\mathbbm{Z}} % Entiers
\newcommand{\Q}{\mathbbm{Q}} % Rationels
\newcommandx{\R}[1][1={}]{\mathbbm{R}^{#1}} % Réels
\newcommandx{\C}[1][1={}]{\mathbbm{C}^{#1}} % Complexes
\newcommandx{\F}[1][1={}]{\mathbbm{F}^{#1}} % Champ quelconque
\newcommand{\M}[3]{\mathbb{M}_{#1\times#2}(#3)}	% Matrices
\newcommand{\Po}[2]{\mathbb{P}_{#1}(#2)} % Polynômes
\newcommand{\Lin}{\mathbb{L}} % Transformations linéaires


% Constantes et autres symboles physiques -----------------------
\newcommand{\eo}{\epsilon_0} % epsilon 0
\renewcommand{\L}{\mathcal{L}}


% Opérateurs
\renewcommand{\H}{\hat{H}}


% Analyse -----------------------
\newcommand{\Res}[2]{\text{Res}\! \p{#1 , #2}} % Résidu de #1 en #2 R{#1,#2}

\newcommand{\bangarang}{$\textbf{B}\alpha\aleph \mathcal{G}\alpha \mathbbm{R}\alpha\aleph\mathcal{G}$}