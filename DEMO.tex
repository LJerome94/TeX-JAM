\documentclass[12pt, a4paper]{article}

\usepackage{mise_en_page}
%----------------------------------------------------------------------
%\usepackage{amssymb} % Mathematical fonts.
%\usepackage{amsfonts} % Mathematical fonts.
\usepackage[nice]{nicefrac} % Nicer fractions
\usepackage{braket} % Dirac Notation.
\usepackage{bbm} % More bold fonts.
%\usepackage{mathrsfs} % Mathematical fonts.
\usepackage{esint} % Integrals
\usepackage{cancel} % Allows to scratch expressions.
\usepackage{mathtools} % Tools for math formating.
\usepackage{slashed} % Allows to slash individual characters.
\usepackage{xargs} % Better handling of optional arguments for commands
%----------------------------------------------------------------------
%\usepackage{lmodern} % Fonts.
\usepackage{feyn} % Feynman Diagrams in mathmode

%%%%%%%%%%%%%%%%%%%%%%%%%%%
% Mathématiques et physique
%%%%%%%%%%%%%%%%%%%%%%%%%%%%
% SI Units -----------------------
% The package 'siunitx' causes unresolved crashes (as of 22/08/31)
\newcommand{\ampere}{\text{A}}
\newcommand{\bell}{\text{B}}
\newcommand{\celsius}{\degree\text{C}}
\newcommand{\coulomb}{\text{C}}
\newcommand{\degree}{\,^{\circ}}
\newcommand{\farad}{\text{F}}
\newcommand{\electro}{\text{e}}
\newcommand{\gram}{\text{g}}
\newcommand{\henry}{\text{H}}
\newcommand{\hertz}{\text{Hz}}
\newcommand{\hour}{\text{h}}
\newcommand{\joule}{\text{J}}
\newcommand{\kelvin}{\text{K}}
\newcommand{\meter}{\text{m}}
\newcommand{\minute}{\text{m}}
\newcommand{\mole}{\text{mol}}
\newcommand{\newton}{\text{N}}
\newcommand{\ohm}{\Omega}
\newcommand{\pascal}{\text{Pa}}
\newcommand{\rad}{\text{rad}}
\newcommand{\second}{\text{s}}
\newcommand{\tesla}{\text{T}}
\newcommand{\torr}{\text{Torr}}
\newcommand{\volt}{\text{V}}
\newcommand{\watt}{\text{W}}
%
\newcommand{\tera}{\text{T}}
\newcommand{\giga}{\text{G}}
\newcommand{\mega}{~\text{M}}
\newcommand{\kilo}{~\text{k}}
\newcommand{\deci}{\text{d}}
\newcommand{\centi}{\text{c}}
\newcommand{\milli}{\text{m}}
\newcommand{\micro}{\mu}
\newcommand{\nano}{\text{n}}
\newcommand{\pico}{\text{p}}
\newcommand{\femto}{\text{f}}
%
\newcommand{\units}[1]{\text{#1}}
\newcommand{\tothe}[1]{\textsuperscript{#1}}
%
\newcommand{\per}{\text{/}}
%
\newcommand{\Time}[3]{#1\hour~#2\minute~#3\second} % TODO Optional arguments.
\newcommand{\Angle}[3]{#1^{\circ}~#2'~#3''} % TODO Optional arguments.


% Better epsilon -----------------------
\let\oldepsilon\epsilon
\let\epsilon\varepsilon
\let\varepsilon\oldepsilon


% Better \bar -----------------------
\renewcommand{\bar}[1]{\mkern 1.5mu\overline{\mkern-1.5mu#1\mkern-1.5mu}\mkern 1.5mu}


% Équations -----------------------
\newcommand{\al}[1]{\begin{align} #1 \end{align}} % Numbered equation(s),
\newcommand{\eqn}[1]{\begin{align*} #1 \end{align*}} % Number-less equation(s),
\newcommand{\sys}[1]{\begin{dcases*} #1 \end{dcases*}} % System of equations.


% Exponents -----------------------
\newcommand{\Exp}[1]{\text{e}^{#1}}		% e^#
\newcommand{\E}[1]{\times 10^{#1}}		% X 10^#


% Delimiters -----------------------
\newcommand{\p}[1]{\left( #1 \right)}	% (#)
\newcommand{\cro}[1]{\left[ #1 \right]}	% [#]
\newcommand{\abs}[1]{\left| #1\right|}	% |#|
\newcommand{\avg}[1]{\left\langle #1 \right\rangle} % <#>
\newcommand{\acc}[1]{\left\lbrace #1 \right\rbrace} % {#}


% Vectors -----------------------
\newcommand{\ve}[1]{\mathbf{#1}} % Upright bold face.
\newcommand{\vu}[1]{\hat{\ve{#1}}} % Hat vector upright bold face
\newcommand{\tens}{\otimes} % Tensor product
\newcommand{\nablav}{\bm{\nabla}} % Bold gradient


% Trig. functions with automatic formating  -----------------------
\newcommandx{\Sin}[2][1={}]{\text{sin}^{#1}\!\p{#2}}
\newcommandx{\Cos}[2][1={}]{\text{cos}^{#1}\!\p{#2}}
\newcommandx{\Tan}[2][1={}]{\text{tan}^{#1}\!\p{#2}}
\newcommandx{\Csc}[2][1={}]{\text{csc}^{#1}\!\p{#2}}
\newcommandx{\Sec}[2][1={}]{\text{sec}^{#1}\!\p{#2}}
\newcommandx{\Cot}[2][1={}]{\text{cot}^{#1}\!\p{#2}}
\newcommandx{\Arcsin}[2][1={}]{\text{arcsin}^{#1}\!\p{#2}}
\newcommandx{\Arccos}[2][1={}]{\text{arccos}^{#1}\!\p{#2}}
\newcommandx{\Arctan}[2][1={}]{\text{arctan}^{#1}\!\p{#2}}
\newcommandx{\Sinh}[2][1={}]{\text{sinh}^{#1}\!\p{#2}}
\newcommandx{\Cosh}[2][1={}]{\text{cosh}^{#1}\!\p{#2}}
\newcommandx{\Tanh}[2][1={}]{\text{tanh}^{#1}\!\p{#2}}


% Matrices -----------------------
\newcommand{\mat}[1]{\begin{bmatrix} #1 \end{bmatrix}} % Matrices with hooks.
\newcommand{\pmat}[1]{\begin{pmatrix} #1 \end{pmatrix}} % Matrices with parentheses.
\newcommand{\deter}[1]{\abs{\begin{matrix} #1 \end{matrix}}} % Determinant.
\newcommandx{\mO}[2][1={}, 2={}]{ \def\temp{#2}\ifx\temp\empty\ve{O}_{#1}\else\ve{O}_{#1\times #2}\fi}% Zero matrix.
\newcommandx{\mI}[2][1={}, 2={}]{ \def\temp{#2}\ifx\temp\empty\ve{I}_{#1}\else\ve{O}_{#1\times #2}\fi}%  Identity matrix.
\newcommand{\Det}[1]{\text{det}\p{#1}} % det(#)
\newcommand{\Tr}[1]{\text{Tr}\p{#1}} % Tr(#)


% Derivatives -----------------------
\newcommand{\D}{\text{d}} % Differential 'd'.
\newcommandx{\dd}[3][1={},3={}]{\frac{\D^{#3}#1}{\D{#2}^{#3}}} % Total derivative according to #2, #1 is the function and #3 is the order.
\newcommand{\del}{\partial} % Partial 'd'.
\newcommandx{\ddp}[3][1={},3={}]{\frac{\del^{#3}#1}{\del{#2}^{#3}}} % Dérivée partielle selon #2, #1 est la fonction est #3 est l'ordre.
\newcommand{\eval}[1]{\left. {#1} \right|} % Bar on the right of expression.
\newcommand{\delbar}{\slashed{\del}} % Partial Inexact differential.
\newcommand{\dbar}{\dj}% Inexact differential.


% Integrals -----------------------
\newcommand{\intinf}{\int\displaylimits_{-\infty}^{\infty}} % From -00 to 00.
\newcommandx{\Int}[2][1={},2={}]{\int\displaylimits_{#1}^{#2}} % Faster bounded integrals.


% Complex numbers -----------------------
\renewcommand{\Re}[1]{\text{Re}\acc{#1}} % Re{#}
\renewcommand{\Im}[1]{\text{Im}\acc{#1}} % Im{#}


% Sets -----------------------
\newcommand{\N}{\mathbbm{N}} % Natural numbers.
\newcommand{\Z}{\mathbbm{Z}} % Integers.
\newcommand{\Q}{\mathbbm{Q}} % Rational numbers.
\newcommandx{\R}[1][1={}]{\mathbbm{R}^{#1}} % Real numbers.
\newcommandx{\C}[1][1={}]{\mathbbm{C}^{#1}} % Complex numbers.
\newcommandx{\F}[1][1={}]{\mathbbm{F}^{#1}} % Some field.
\newcommand{\M}[3]{\mathbb{M}_{#1\times#2}(#3)}	% Matrices.
\newcommand{\Po}[2]{\mathbb{P}_{#1}(#2)} % Polynomials.
\newcommand{\Lin}{\mathbb{L}} % Linear maps.


% Constants and physical symbols -----------------------
\newcommand{\eo}{\epsilon_0} % epsilon 0.
\renewcommand{\L}{\mathcal{L}} % Lagrangian.
\usepackage{references}
\usepackage{special}

%%%%%%%%%%%%%%
% Page titre %
%%%%%%%%%%%%%%
\title{Titre du travail}
\author{\JL} %Change pour votre ou vos noms
\teacher{Professeur}
\class{Titre du cours}
\date{\US\\ \today} % Changer ici pour un établissement ou date différente

\setlength{\parskip}{1em}

%%%%%%%%%%%%%%%%%%%%%%%%%
% Commencer le document %
%%%%%%%%%%%%%%%%%%%%%%%%%
\begin{document}

% Faire la page titre
\maketitlepage

%--------------------------------------------------------------------------------------------
\section{Ceci est une section}

\subsection{Ceci est une sous-section}

\subsubsection{Ceci est une sous-sous-section}

On peut écrire des équations sans numéros:
\eqn{
    E=mc^2
}
On peut aussi en écrire avec des numéros mais pas à toutes les lignes:
\al{
    F&=ma\label{eq:loi_de_newton}\\
    ma&=F\notag\\
    a&=\frac{F}{m}
}
Que l'on peut référer \eqref{eq:loi_de_newton}

On peut aussi faire des système d'équation
\eqn{
    \sys{
        x = 1\\
        y = 2\\
        z = 3
    }
}
On peut juxtaposer des équations (fonctinne seulement avec \textit{al\{\}}):
\al{
    \begin{split}
        Av=\lambda v
    \end{split}
    \begin{split}
        \ddot{\psi}=-\omega^2\psi
    \end{split}\notag
}

\bigbox{
On peut faire des section dans une boîte et faire des équation elles-aussi encadrées:
\eqn{
    \boxed{\psi(t)=A\cos{\omega t}+B\sin{\omega t}}
    }
}

On peut \yellow{aussi} \purple{mettre} \blue{du} \orange{texte} \green{en} \red{couleur}

On peut faire des listes:
\begin{itemize}
    \item premier
    \item deuxième
    \item troisième
\end{itemize}

Et des listes énumérées:
\begin{enumerate}
    \item premier
    \item deuxième
    \item troisième
\end{enumerate}

On a aussi le formatage automatique de plusieurs encadrés:
\eqn{
    \p{\frac{A}{B}} \cro{\frac{C}{D}} \norm{\frac{E}{F}} \avg{\frac{G}{H}} \acc{\frac{I}{J}}
}
Ce formatage s'applique aussi aux rapport trigonométriques:
\eqn{
    \Sin{\frac{A}{B}}~\Atan{\frac{B}{C}}~\Re{\frac{y}{x}}~\Im{z}
}

On peut écrire des vecteurs de divers façons:
\eqn{
    \Vec{x}~\ve{x}~\hat{x}~\vu{x}
}
On a certain raccourci pour les ensembles:
\eqn{
    \N~\Z~\R~\C
}
Sinon nous avons plusieurs polices:
\eqn{
    \mathbf{A B C D E F G H I J K L M N O P Q R S T U V W X Y Z}\\
    \mathcal{A B C D E F G H I J K L M N O P Q R S T U V W X Y Z}\\
    \mathbb{A B C D E F G H I J K L M N O P Q R S T U V W X Y Z}\\
    \mathbbm{A B C D E F G H I J K L M N O P Q R S T U V W X Y Z}\\
    \mathfrak{A B C D E F G H I J K L M N O P Q R S T U V W X Y Z}\\
    \mathscr{A B C D E F G H I J K L M N O P Q R S T U V W X Y Z}
}
On peut facilement mettre des figures comme suit:
% Les arguments sont {fichier image}{taille}{légende}{label}
\Figure{Demo/graphique.png}{0.5}{Ceci est une légende.}{fig:graphique}

On peut réfèrer aux figures (comme ceci \ref{fig:graphique}).

Les prochaines équations montres divers racourcis mathématiques:
\al{
    \begin{split}
        A=\mat{
        a & b \\
        c & d
        }=
        \pmat{
        a & b \\
        c & d
        }
    \end{split}
    \begin{split}
        \Det{A}=\deter{
        a & b \\
        c & d
        }
    \end{split}\notag
}
\eqn{
    (4\ee{12})~\Exp{\gamma}~\Expi{\omega t+\phi}~\Expmi{\omega t+\phi}\\
    \D~\dd{\phi}{r}~\ddn{3}{\phi}{r}~\dx{f}~\dy{f}~\dt{f}~\del~\ddp{f}{x}~\ddpn{3}{f}{y}~\eval{\dd{\phi}{r}}_{1}\\
    \int~\iint~\iiint~\oint~\oiint~\ointctrclockwise~\varointclockwise
}
Et pour les unités de mesures (ces lignes sont plus sensées si on voits le code):
\eqn{
    \ampere~\celsius~\coulomb~\degree~\gram~\henry~\hertz~\hour~\farad~\kelvin~\m~\joule~\minute~\mole~\newton~\ohm~\rad~\second~\volt~\watt\\
    \giga\m~\mega\m~\kilo\m~\deci\m~\centi\m~\milli\m~\micro\m~\nano\m~\pico\m\\
    \m\per\second\\
    \Time{2}{30}{46}
}

Ensuite des constantes, symboles ou opérateurs physique:
\eqn{
    \eo~\K~\fourier{f}~\lagrange
}

On peut annuler des termes:
\eqn{
    \xcancel{\Sin{\frac{Y+X}{Z}}}~~~\cancelto{1}{\Exp{\frac{D+X}{y}}}~~~\cancel{\frac{C}{AB}}
}

\partie{On peut avoir des titre surlignés}

% Importer le code d'un autre fichier
\subfile{Demo/sous_fichier}

%%%%%%%%%%%%%%%
% Situationel %
%%%%%%%%%%%%%%%
On peut présenter du code python avec des sections comme suit (requiert une écriture particulière):
% Code
\begin{lstlisting}[language=Python]
import numpy as np

def func(x,y):
    if y == True:
        return 4*x+3
    
print("Hello world")
\end{lstlisting}

On peut faire directement de graphiques dans \LaTeX{}:
% Graphique
\begin{figure}[H]
	\centering
	\begin{tikzpicture}
	\begin{axis}[
		my axis style,
		width=0.8\textwidth,
		height=.5\textwidth,
		legend entries={
			$y = x\Exp{-x}$,
			$y = 2x\Exp{-x}$,
			$y = \frac{x}{2}\Exp{-x}$
		},
		legend pos=north east
	]
	
	\addplot[
		domain=0:5,
		thick,
		-
	]
	{x*exp(-x)};

	\addplot[
		domain=0:5,
		thick,
		red,
		dashed,
		-
	]
	{2*x*exp(-x)};

	\addplot[
		domain=0:5,
		thick,
		blue,
		dashdotted,
		-
	]
	{.5*x*exp(-x)};
	
	\fill[
		black
	]
	(1,.36788) circle (2pt) node[above right] { $(1, 1/\Exp{})$};
	
	\end{axis}
	\end{tikzpicture}
	\caption{Ceci est un graphique.}
	\label{fig:my-awesome-graph}
\end{figure}

On peut faire toute sorte de diagrammes:
\al{
    \feynmandiagram [horizontal=a to b] { % Feynman
        i1 -- [fermion] a -- [fermion] i2,
        a -- [photon] b,
        f1 -- [fermion] b -- [fermion] f2,
    };
    \\
    \begin{circuitikz} \draw %Circuit électrique
    (0,0) to[battery] (0,4)
        to[ammeter] (4,4) -- (4,0) -- (3.5,0)
        to[lamp, *-*] (0.5,0) -- (0,0)
    (0.5,0) -- (0.5,-2)
        to[voltmeter] (3.5,-2) -- (3.5,0)
    ;
    \end{circuitikz}
    \\
    \begin{circuitikz} \draw % Circuit logique
    (0,2) node[and port] (myand1) {}
    (0,0) node[and port] (myand2) {}
    (2,1) node[xnor port] (myxnor) {}
    (myand1.out) -- (myxnor.in 1)
    (myand2.out) -- (myxnor.in 2);
    \end{circuitikz}
    \\
    \chemfig{A*6(-B=C(-CH_3)-D-E-F(=G)=)}
}

%Échecs
\newgame
\mainline{1.e4}    
\showboard


Plusieurs symboles sont disponibles:
\faAndroid \faAmazon\faAreaChart \faAssistiveListeningSystems \faBank \faBellSlashO \faCab \faChrome \faCut \faFilePdfO ~ne sont que quelques exemples.

Aussi des touches de calviers:
\eqn{
    \Return~\Spacebar~\Ctrl~\Shift~\keystroke{F1}~\keystroke{A}
}

%%%%%%%%%%%%%%%%%%%%%%%%
% Inutiles mais drôles %
%%%%%%%%%%%%%%%%%%%%%%%%
% Pour utiliser cette partie, enlever des commentaires les packages dans "mise_en_page.tex".
Maintenant pour le plaisir:
\eqn{
    \pumpkin~\skull~\mathbat~\mathghost~\mathcloud~\mathwitch~\bigskull~\bigpumpkin\\
    \fullmoon~\newmoon~\leftmoon~\rightmoon~\astrosun~\mercury~\earth~\jupiter~\saturn ~\uranus~\neptune~\pluto\\
    \aries~\taurus~\gemini~\cancer~\leo~\virgo~\libra~\scorpio~\sagittarius ~\capricornus~\aquarius~\pisces~\\
    \textpmhg{A B C D E F G H I J K L M N O P Q R S T U V W X Y Z 1 2 3 4 5 6 7 8 9 0}\\
}
\EOofficerI\EOSaw\EOPatron\EOScorpius\EORain\EOxvii \EOpi\EOki\EOSa

% Ajouter une page blanche
\blankpage

%--------------------------------------------------------------------------------------------
%%%%%%%%%%%%%%%%%%%%%%%%
% Terminer le document %
%%%%%%%%%%%%%%%%%%%%%%%%

\end{document}